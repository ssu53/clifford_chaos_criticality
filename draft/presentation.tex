\documentclass[9pt]{beamer}
% \documentclass[handout,9pt]{beamer}
\usetheme{Antibes}
\usecolortheme{beaver}
% \setbeamertemplate{headline}[default]
\addtobeamertemplate{navigation symbols}{}{%
    \usebeamerfont{footline}%
    \usebeamercolor[fg]{footline}%
    \hspace{1em}%
    \insertframenumber/\inserttotalframenumber
}

\useoutertheme[subsection=false]{miniframes}
\setbeamertemplate{itemize item}{\tiny\color{black}$\blacksquare$}

\setbeamersize{text margin left=0.7cm, text margin right=0.7cm} 

\usepackage{amsmath}   
% \usepackage{amsfonts}
\usepackage{physics}    
% \usepackage{graphicx}
\usepackage{subfigure}
%\usepackage{caption}
% \usepackage{verbatim}
% \usepackage{float}
%\usepackage{svg}

\usepackage{csquotes}
% \usepackage{slashed}

\setlength\parindent{0pt}
\setlength\parskip{5pt}

\usepackage{makecell}

\renewcommand{\arraystretch}{1.5}



% \newcommand\numberthis{\addtocounter{equation}{1}\tag{\theequation}}
% \DeclareMathOperator{\lagr}{\mathcal{L}}
% \DeclareMathOperator{\ham}{\mathcal{H}}


\addtolength{\skip\footins}{1pc plus 3pt}

\title{\huge Chaos and Measurement-Induced Criticality on Stabiliser Circuits}
\subtitle{}
\author[1]{\large Shiye Su\inst{1}\\ \quad Advisors: Sarang Gopalakrishnan,\inst{2} David Huse\inst{1}}
\institute{\inst{1} Department of Physics, Princeton University \\
		   \inst{2} The City University of New York}
%\date{\today}
\date{}


\begin{document}



\begin{frame}[plain]
\titlepage
\end{frame}


\section{L'Ordre du Jour}

\begin{frame}
\frametitle{L'Ordre du Jour}

{\centering
Does the system thermalise? \\
How does this look in terms of entanglement? RMT? \\
}

\pause
\vspace{0.5cm}
Setting: random quantum circuits
\begin{itemize}
\item Preliminaries
\item Measurement-induced entanglement transition
\item Clifford spectral statistics
\end{itemize}



\end{frame}

% my thesis is situated within the study of non-equilibrium many body physics. the questions we're interested in are for example, will a system evolving under certain conditions thermalise? how is this behaviour reflected in the entanglement growth, and signatures from random matrix theory? it turns out that a system evolving unitarily, subject to repeated projective local measurements, exhibits a critical transition: phase that between weak and strong entanglement growth - we'll make this statement more precise later. this is called the measurement-induced criticality. 

\section{Preliminaries}
\subsection{}

\begin{frame}
\frametitle{Pauli group and stabilisers}

Pauli group
\begin{equation}
\mathcal{P}_1 \equiv \{\pm I, \pm i I, \pm X, \pm i X, \pm Y, \pm i Y, \pm Z, \pm i Z\}.
\end{equation}

\begin{equation}
\mathcal{P}_n = \mathcal{P}_1^{\otimes n}
\end{equation} 

% \pause
% $\{\mathcal{P}_n / U(1)\}$ is a basis for $(\mathbb{C}^{2\times2})^{\otimes n} = \mathbb{C}^{2^n\times2^n}$ 

\pause
\vspace{0.5cm}
Stabiliser set $S \subset \mathcal{P}_n$ for state subspace $V_S$
\begin{equation}
V_S = \{\ket{\psi} \ | \ s \ket{\psi} = \ket{\psi} \quad \forall \ s \in S \}
\end{equation}

$n$-qubit stabiliser state (mod phase) is fully specified by $n$ Hermitian stabilisers.

\end{frame}



\begin{frame}
\frametitle{Clifford gates}

Normaliser in the unitaries of $\mathcal{P}_n$  
\begin{equation}
C_n = \{U \in U_{2^n} \ | \ U^\dagger \mathcal{P}_n U = \mathcal{P}_n\} / U(1)
\end{equation}

\pause
Gottesman-Knill Theorem
\begin{displayquote}
Quantum computation involving only Clifford gates and measurements of Pauli group observables on stabiliser states can be classically simulated in polynomial time.
\end{displayquote}
% (Pauli strings have $\mathbb{Z}_2$ representation.)

\pause 
Qualitatively similar dynamics to Haar unitaries. Unitary 3-Design. 
% For $t\leq3$ but not $t\geq4$
% \begin{equation}
% \frac{1}{K} \sum_{k=1}^K P_{(t,t)}(U_k) = \int_{\mathcal{U}_d} \dd{U} P_{(t,t)} (U)
% \end{equation}

\end{frame}

% Most generic requirement of quantum gate: unitarity.
% We also want: stabiliser states map to other stabiliser states

% Clifford gates reproduce Haar unitary averages of the $t$-th moment, for $t \leq 3$ (but not $t\geq4$).

% Aaronson and Gottesman algorithm
% pauli string multiplication implemented by binary addition, nonlinear phase update
% won't discuss these algorithms for gate updates, measurements, and entanglement entropy





\begin{frame}
\frametitle{Bipartite entanglement entropy}

\begin{figure}
\subfigure[OBC.]{\includegraphics[width=0.5\textwidth]{fig_cut_obc.pdf}}
\subfigure[PBC.]{\includegraphics[width=0.3\textwidth]{fig_cut_pbc.pdf}}
\end{figure}

\pause

Renyi entropies
\begin{equation}
S_\alpha(\rho) \equiv \frac{1}{1-\alpha} \log \trace (\rho^\alpha)
\end{equation}

Degenerate spectrum for stabiliser circuits!


\end{frame}

% von Neumann entropy: $S_1 \equiv \lim_{\alpha\rightarrow1} S_\alpha(\rho)$



\section{Measurement-induced criticality}
\subsection{}

\begin{frame}
\frametitle{The Vanilla MIC}

\begin{figure}
\centering
\includegraphics[trim={0, 0, 0, 1cm}, clip, width=0.5\textwidth]{fig_brick.pdf}
\pause
\includegraphics[trim={1cm 1cm 1cm 1.5cm}, clip, width=0.5\textwidth]{fig_ent1D.pdf}
\end{figure}

{\centering
$p<p_c$ volume law \\
$p>p_c$ area law \\
}

% \begin{itemize}
% \item<+-> Critical phases: $p<p_c$ volume law, $p>p_c$ area law
% \item<+-> Percolation mapping
% \item<+-> Other probes of the transition
% \end{itemize}

\end{frame}

% phenomena for Haar unitary gates, also MIC is seen for Cliffords
% initialise product state, measure prob p
% exact for Haar random gates. $S_0$ is the minimum cut.
% pc approx. 0.158
% DATA COLLAPSE


\begin{frame}
\frametitle{Modelling delayed measurements}

\begin{figure}
\centering
\includegraphics[width=0.6\textwidth]{fig_ancilla.pdf}
\end{figure}

\pause
$U$ is random Clifford layer; $V$ is CNOT ancilla coupling (prob. $p$)
\begin{equation}
K_{\vec{m}} = \mathcal{T} \bigg[ \prod_{t=1}^\tau \bigg(V_t U_t V_{t-\frac{1}{2}} U_{t-\frac{1}{2}}\bigg) \bigg]
\end{equation}

\pause
Quantum channel
\begin{equation}
\rho \rightarrow
\sum_{\vec{m}} K_{\vec{m}} \bigg(\rho \otimes \dyad{\vec{m}(0)}\bigg) K_{\vec{m}}^\dagger 
= \sum_{\vec{m}} \rho_{\vec{m}} (\tau) \otimes \dyad{\vec{m}(\tau)}
\end{equation}
Ancilla projection recovers the usual MIC.

\end{frame}


\begin{frame}
\frametitle{Entanglement localisation}

No critical behaviour at $p_c$

\pause
Consider $L/3$ mutual information on open chain
\begin{equation}
I(A:B) = S(A) + S(B) - S(AB)
\end{equation}
\begin{figure}
\centering
\includegraphics[trim={0.5cm 0.7cm 1cm 1cm}, clip, width=0.5\textwidth]{fig_ancL3_96.pdf}
\end{figure}

% \pause
% \begin{itemize}
% \item Long-range entanglement never develops for some $p$. 
% \item Projecting ancillae migrates entanglement non-monotonically.
% \end{itemize}

\end{frame}

% showed for a variety of system sizes


\begin{frame}
\frametitle{Correlation length}

% How are correlations spatially localised in the pre-measurement state?
% \begin{figure}
% \centering
% \includegraphics[trim={19cm 16cm 3.5cm 2cm}, clip, width=0.5\textwidth]{fig_corlen.pdf}
% \end{figure}

Spatial correlation with $L$-independent correlation length $\xi \propto p^{-\nu}$.
\begin{figure}
\centering
\begin{minipage}{0.55\linewidth}
\includegraphics[trim={1cm 0 2cm 0}, clip, width=1\textwidth]{fig_corexp.pdf}
\end{minipage}
\hspace{0.5cm}
\begin{minipage}{0.35\linewidth}
\begin{tabular}{l l}
\hline
system size & $\nu$ \\
\hline
$L=48$ 	& 0.632 \\
$L=72$ 	& 0.725 \\
$L=96$ 	& 0.753 \\
$L=144$ & 0.779 \\
$L=192$ & 0.789 \\
\hline
\end{tabular}
\end{minipage}
\end{figure}

\end{frame}

% Pre-measurement mutual information between end regions $A$ and $B$ on an open chain, as a function of their separation.

% mutual information increases linearly time-law for separations less than $\xi(p)$, but the `elbow' marking $\xi(p)$ remains constant.

% correlation length as function of $p$


\section{Clifford spectral statistics}
\subsection{}

\begin{frame}
\frametitle{RMT signatures of quantum chaos}

Do Cliffords exhibit these signatures?

\pause
Chaotic quantum systems have spectral statistics described by RMT at late times (BGS conjecture).

\pause
\vspace{0.5cm}
Floquet $U$ models $H(t) = H(t+\tau)$.
\begin{equation}
U = e^{-iH}
\end{equation}
Study level statistics of $H$.

\pause
\vspace{0.5cm}
\begin{itemize}
\item<+-> nearest level spacing 
\item<+-> spectral form factor
\end{itemize}

\end{frame}

% nonlocality. onset with Thouless time
% periodically driven systems
% Poissonian vs. Wigner-Dyson. not that useful because level degeneracies

\begin{frame}
\frametitle{Spectral form factor}

The energy density function
\begin{equation}
\rho(E) = \sum_{n=1}^N \delta(E-E_n)
\end{equation}

has mean two-point correlation in Fourier space
\begin{equation}
K(t) 
= \frac{1}{N^2} \sum_{m,n} \ev{ e^{-i(E_m-E_n)t} }
= \ev{ \trace{U(t)} \trace{U^\dagger(t)} }
\end{equation}

\pause
\emph{Cliffords.} Let $G(U(t))$ be Pauli strings that generate all $P \in \mathcal{P}_L$ such that 
\begin{equation}
P \rightarrow_{U(t)} P \mod U(1)
\end{equation}

We show
\begin{equation}
K(t) = 
\begin{cases}
2^{\abs{G(U(t))}}, & \text{if } P \rightarrow_{U(t)} +P, \ \forall P \in G(U(t)) \\
0, 				   & \text{otherwise}
\end{cases}
\end{equation} 

% Polynomial time to find $G(U(t))$ and then to check their phase!

\end{frame}



\begin{frame}
\frametitle{Spectral form factor}

\begin{minipage}{0.3\linewidth}
\begin{equation}
K_\text{CUE}(t) = 
\begin{cases}
N^2 & t=0 \\ 
t 	& 0 < t \leq N \\ 
N 	& N \leq t
\nonumber
\end{cases}
\end{equation}
\end{minipage}
\hspace{0.7cm}
\begin{minipage}{0.6\linewidth}
\onslide<2,3>{
\begin{equation}
K_\text{Clif.}(t)
\begin{cases}
\text{exponential ramp, sub-linear ramp time} \\
\text{late-time $K(t) \propto 2^{cL}$ for some $c>1$} \\
\text{small stabiliser orbits}
\end{cases}
\nonumber
\end{equation}
}
\end{minipage}



\begin{figure}
\centering
\subfigure{\includegraphics<2,3>[trim={3cm 0 3cm 0}, clip, width=0.7\textwidth]{fig_sff_L16.pdf}}
\subfigure{\includegraphics<3>[trim={2.5cm 0 3cm 0}, clip, width=0.7\textwidth]{fig_sff_ft.pdf}}
\end{figure}

% \begin{table}
% \centering
% \begin{tabular}{c c c}
% \hline
% \thead{system size} &
% \thead{late time \\ $K(t)$} & 
% \thead{late time \\ $\ev{\abs{G(U(t))}}$} \\
% \hline
% $L=8$ 	& $2^{10.40}$ & 2.73 \\
% $L=16$ 	& $2^{20.40}$ & 4.09 \\
% $L=24$ 	& $2^{32.91}$ & 5.70  \\
% $L=32$ 	& $2^{45.93}$ & 7.39 \\
% $L=48$ 	& $2^{56.86}$ & 10.88 \\
% $L=64$ 	& $2^{83.82}$ & 14.35 \\
% \hline
% \end{tabular}
% \end{table}


\end{frame}

% broken time-reversal symmetry


\section{Outlook}

\begin{frame}
\frametitle{Outlook}

...Past
\begin{itemize}
\item Extension of MIC to including measurement ancillae
\item Clifford spectral statistics
\end{itemize}

\vspace{0.5cm}
\pause
Future...
\begin{itemize}
\item Time correlations 
\item Spectral statistics to capture measurement transition
\item Higher dimensions
\end{itemize}

\end{frame}





% \begin{frame}

% {\tiny
% \nocite{*}
% \setbeamertemplate{bibliography item}[text]
% \bibliographystyle{h-physrev}
% \bibliography{bib}
% }

% \end{frame}



\end{document}